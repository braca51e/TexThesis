%\begingroup
%\let\cleardoublepage\clearpage


% English abstract
\cleardoublepage
\chapter*{Abstract}
%\markboth{Abstract}{Abstract}
\addcontentsline{toc}{chapter}{Abstract} % adds an entry to the table of contents
% put your text here
%\justify
Compressed sensing (CS) is a novel signal processing theory stating that a signal can be fully recovered from a number of samples lower than the boundary specified by Nyquist–Shannon sampling theorem, as long as certain conditions are met. In compressed sensing the sampling and compression occur at the same time. While that allows to have signals sampled at lower rates, it creates the necessity to put more workload on the reconstruction side. Most algorithms that are used for recovering the original signal are called iterative, that is because they solve an optimization problem that is computationally expensive. Not only that, but in some cases the reconstruction does not have good quality. This thesis proposes a non-iterative method based on a Deep Learning (DL) framework to recover signals from CS samples in a faster way while maintaining a reasonable quality. DL has already proved its potential in different image applications. As a result, this approach is tested using grayscale images and recent DL software packages. The results are compared against iterative methods in terms of the amount of time needed for full reconstruction as well as the quality of the reconstructed image. The experiments showed both the effectiveness of this method for speeding up the recovery process and also the quality was maintained at a considerable quality level.  
%\lipsum[1-2]
\vskip0.5cm
%put your text here


%% German abstract
%\begin{otherlanguage}{german}
%\cleardoublepage
%\chapter*{Zusammenfassung}
%%\markboth{Zusammenfassung}{Zusammenfassung}
%% put your text here
%\lipsum[1-2]
%\vskip0.5cm
%Stichwörter: 
%%put your text here
%\end{otherlanguage}




%% French abstract
%\begin{otherlanguage}{french}
%\cleardoublepage
%\chapter*{Résumé}
%%\markboth{Résumé}{Résumé}
%% put your text here
%\lipsum[1-2]
%\vskip0.5cm
%Mots clefs: 
%%put your text here
%\end{otherlanguage}


%\endgroup			
%\vfill
