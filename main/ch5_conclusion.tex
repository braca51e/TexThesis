\chapter{Conclusion}
%Here you can see a citation: \cite{Donoho01}.
In this thesis we have proposed a method that, after evaluation of our experiments, succesfully recovers original images from compressed measurements. We have found out that this method is able to perform much faster and improves quality and visual impact when compared to traditional methods. Namely, we have introduced an approach that uses deep learning with CNN's. Two different network architectures have been proposed: small network and large network. Small network has a performance that is not as good as the large network but, it has the advantage of being lighter and thereby perfornming twice as fast. Besides, the loss in reconstruction quality is not large. Large network is deeper and has more parameters, but its performance is better. In fact, it produces images that are 2 to 4 dB's higher in terms of PSNR when compared to iterative state-of-the-art methods publicly available. Not only that but, our method seems to be more capable of encapsulating and reconstructing the original structure of the image which is supported by the higher values of SSIM. Having high values of SSIM provides a better visual impact. Besides, if we decrease the compresion ratio our network gives images that without further processing seem appealing to the human eye. That is, no extra denoising phase is needed. 
\newline \newline 
Based on our experimentation we have also demonstrated that while learning the sensing matrix $\Phi$ imposes extra design constraints, it also improves considerably reconstructed images as compraed to i.i.d. Gaussian matrices. Furtheremore, learning the sensing matrix $\Phi$ may avoid generating random matrices in hardware, which is a very expensive process.
\newline \newline
Even though our approach is faster because it makes used of optimized GPU's, it may still be a good choice for many real time applciations due to the fact that iterative methods are very slow. With this approach we can give a speed up of up to 188x using the heaviest large network. Using the other networks would increse that ratio even more.    


\section{Future work}
There are interesting ideas still remaining to be tested. First, we believe that increasing and extending the number of training images will ultimately generate better results. Therefore, changing the training dataset for a much bigger one seems to be promissing. Second, trying to reconstruct images with higher reslution like 4K is also good way to asses our implementation. We think that our approach is not sensible to image size but having real proof is also necessary to support this claim. In fact, there is some evidence that scaling the image size produces even better results. Third, having a fixed sensing matrix $\Phi$ is also a good improvement but another thing to consider is binarizing it.  That is, the elements of the sensing matrix only have values 0 or 1. That would allow even cheaper hardware implementation for CS sensors. Recently, there has been development on this topic. Integrate batch normalization during the training phase also help improve the reconstruction of the images but it stil needs to be further investigated.




