\chapter{Conclusion}
%Here you can see a citation: \cite{Donoho01}.
In this thesis we have proposed a method that, after evaluation, succesfully recovers original images from compressed measurements. We have found out that this method is able to perform much faster and improves quility and visual impact when compared to traditional methods. Namely, we have introduced an approach that uses deep learning with CNN's. Two different network architectures have been proposed: alpha network and beta network. Alpha network has a performance that is not as good as the beta network but, it has the advantage of being lighter and thereby perfornming twiece as faster. Besides, the loss in reconstruction quality is not so big. Beta network is deeper and has many more parameters, but its performance is better. In fact, it produces images that are 2 to 4 dB's higher in terms of PSNR when compared to iterative state-of-the-art methods publicly available. Not only that but, our method seems to be more capable of encapsulating and reconstructing the original structure of the image which is supported by the higher values on SSIM. Having high values on SSIM provides a better visual impact.
\newline \newline 
Based on our experimentation we have also demonstrated that while learning the sensing matrix $\Phi$ imposes extra design constraints, it also improves considerably reconstructed images as compraed to i.i.d. Gaussian matrices. Furtheremore, learning the sensing matrix $\Phi$ may avoid generating random matrices in harware, which is a very expensive process.
\newline \newline
Even though our apprach is faster because it makes used of optimized GPU's, it may still be a good choice for many real time applciations due to the fact that iterative methods are very slow. With this approach we can give a speed up of up to 6000x using the heaviest beta network. Using the other networks would increse that ratio even more.    


\section{Future work}
There are interesting ideas still remained to be tested. First, we believe that increasing and extending the number of training images will ultimately generate better results. Therefhore, changing the training dataset for a much bigger one seems to be primissing. Second, trying to recosntruct images with higher reslution like 4K is also good way to asses our implementation. We thinks that our approach is not sensible so image sizes but having real proof is also necessary to support this claim. Third, having a fixed sensing matrix $\Phi$ is also good improvements but another thing to consider is binarize it.  That is, the elements of the matrix or only 0 or 1. That would allow even cheaper harware implementation for CS sensors. Recently, there has been development on this topic.    




