\chapter{Conclusion}
%Here you can see a citation: \cite{Donoho01}.
In this thesis we have proposed a method that succesfully recovers original images from compressed measurements that is able to perform much faster and improves quility and visual impact when compared to traditional methods. Namely, we have introduced an approach that uses deep learning with CNN's. Two different network architectures have been proposed: alpha network and beta network. Alpha network has a performance that is not as good as the beta network but, it has the advantage of being lighter and thereby perfornming twiece as faster. Besides, the loss in reconstruction quality is still not
 It is also important to point out that both networks performs better than any state-of-the-art methods publicly available. In particular, beta network perfoms faster and produces images that are 2 to 4 dB's better in terms of PSNR when compared to iterative methods.   \
We have also demosntrated that while learning the sensing matrix $\Phi$ imposes extra design constraints, it also improves considerably reconstructed images as compraed to i.i.d. Gaussian matrices. Not only that but, learning the sensing matrix $\Phi$  



our approach is able to encapsulate the structure of the image
\section{Future work}

-extend texting to 4K
-learn phi by binarizing the weights
-increase the number of images for training



